\documentclass[12pt,a4paper,titlepage, oneside]{book}
\usepackage[utf8]{inputenc}
\usepackage[russian]{babel}
\usepackage[OT1]{fontenc}
\usepackage{amsmath}
\usepackage{amsthm}
\usepackage{mathrsfs}
\usepackage{indentfirst}
\usepackage{amsfonts}
\usepackage{amssymb}
\usepackage[left=2cm,right=2cm,top=2cm,bottom=2cm]{geometry}
\title{Функциональный анализ}

\newcommand{\overbar}[1]{\mkern 1.5mu\overline{\mkern-1.5mu#1\mkern-1.5mu}\mkern 1.5mu}

\theoremstyle{definition}
\newtheorem*{definition}{Определение}

\theoremstyle{plain}
\newtheorem*{theorem}{Теорема}

\theoremstyle{remark}
\newtheorem*{remark}{Замечание}

\theoremstyle{remark}
\newtheorem*{example}{Пример}

\theoremstyle{remark}
\newtheorem*{examples}{Примеры}

\theoremstyle{remark}
\newtheorem*{cexample}{Контр-пример}

\theoremstyle{plain}
\newtheorem*{lemma}{Лемма}

\theoremstyle{plain}
\newtheorem*{corollary}{Следствие}

\setcounter{tocdepth}{1}

\def\labelitemi{--}

\renewcommand{\qedsymbol}{\rule{0.7em}{0.7em}}

\begin{document}

\begin{titlepage}

\begin{center}
\vfill

Санкт-Петербургский государственный университет\\
\ \\

\vfill

{\large\bf ФУНКЦИОНАЛЬНЫЙ АНАЛИЗ\\}
\ \\
Лекции для студентов факультета ПМ-ПУ\\
(III курс, 6-ой семестр)

\vfill

\hfill\vbox
{
\hbox{Доцент кафедры моделирования электромеханических}
\hbox{и компьютерных систем, кандидат физ.-мат. наук}
\hbox{Владимир Олегович Сергеев}
}

\vfill

Санкт-Петербург, 2016
\end{center}

\end{titlepage}


\section{Сопряженный оператор}
Пусть оператор $A \subset L(X,Y)$, функционал $f \in Y^*$. Вычислим значения $<Ax,f>$. Эти значения можно трактовать как значения функционала $\phi$, определённого на элементах пространства $X$:
\begin{equation}
<Ax,f>=\phi(x)
\end{equation}
Ясно, что функционал $\phi$ задан на всём пространстве $X$ и зависит от выбора функционала $f$:
\begin{enumerate}
\item функционал $\phi$ аддитивен и однороден:
\begin{center}
$\phi(\lambda_1x_1+\lambda_2x_2) = 
<A(\lambda_1x_1+\lambda_2x_2),f> =
\lambda_1<Ax_1,f> + \lambda_2<Ax_2,f> =$\\$
= \lambda_1\phi(x_1) + \lambda_2\phi(x_2)$
\end{center}
\item функционал $\phi$ линейный:
\begin{center}
$\lVert \phi \lVert \leq \underset{\lVert x\lVert =1}{\sup} 
\vert \phi(x) \vert \leq  \underset{\lVert x\lVert =1}{\sup} 
\lVert Ax \lVert \cdot \lVert f \lVert =
 \lVert A \lVert \cdot \lVert f \lVert $
\end{center}
\end{enumerate}
Таким образом равенство $<Ax,f> = <x,\phi>$ определяет линейный функционал $\phi \in X^*$:
\begin{itemize}
\item каждому $f\in Y^*$ сопоставляется линейный функционал
 $\phi\in X^*$, зависящий от выбора оператора $A$: эту зависимость обозначим $\phi=A^*\cdot f$.
\end{itemize}
Эта зависимость от $A$ однородна и аддитивна:
если $A=\alpha_1A_1+\alpha_2A_2$, то 
\begin{center}
$\phi=<\alpha_1A_1+\alpha_2A_2> = \alpha_1<A_1,f>+\alpha_2<A_2,f> =
\alpha_1A_1^*f+\alpha_2A_2^*f$
\end{center}
Оператор $A^*$, определённый на $Y^*$ линеен: согласно неравенству:
\begin{center}
$\lVert \phi \lVert = \lVert A^* \cdot f \lVert \leq \lVert A \lVert \cdot \lVert f \lVert$
\end{center}
\begin{center}
${\lVert A^* \lVert}_{Y^* \to X^*} \leq 
{\lVert A \lVert}_{X \to Y}$ и $A^* \in L(Y^*,X^*)$
\end{center}
Докажем, что $\lVert A^* \lVert = 
\lVert A \lVert$. Возьмём произвольный элемент $x_0 \in X$ и вычислим $y_0=Ax_0$. Для элемента $y_0$ по теореме Хана-Банаха существует линейный функционал $f$, такой что
$f(y_0)= \lVert y_0 \lVert$, $\lVert f \lVert =1$.\\
Тогда
\begin{center}
$\lVert Ax_0 \lVert = \vert <Ax_0,f> \vert =
\vert <x_0,A^*f> \vert \leq 
\lVert x_0 \lVert \cdot \lVert A^* \lVert
\cdot \lVert f \lVert = \lVert A^* \lVert
\cdot \lVert x_0 \lVert$
\end{center}
Следовательно $\lVert A \lVert \leq
\lVert A^* \lVert$ и вместе с оценкой $\lVert A^* \lVert \leq 
\lVert A \lVert$ получаем $\lVert A^* \lVert = 
\lVert A \lVert$.
\begin{definition}
Линейный оператор $A^* \in L(Y^*,X^*)$ называется оператором, сопряжённым с оператором $A \in L(X,Y)$, если для любого $f \in Y^*$ выполнено:
\begin{center}
$<Ax,f> = <x,A^*f>$
\end{center}
\end{definition}
\begin{remark}
Ясно, что $\lVert A^* \lVert = 
\lVert A \lVert$.
\end{remark}
\end{document}
